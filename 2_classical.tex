\documentclass{tufte-book}

\hypersetup{colorlinks}% uncomment this line if you prefer colored hyperlinks (e.g., for onscreen viewing)

%%
% Book metadata
\title{Physics II - Classical Mechanics}
\author{Ragnamus}

%%
% If they're installed, use Bergamo and Chantilly from www.fontsite.com.
% They're clones of Bembo and Gill Sans, respectively.
%\IfFileExists{bergamo.sty}{\usepackage[osf]{bergamo}}{}% Bembo
%\IfFileExists{chantill.sty}{\usepackage{chantill}}{}% Gill Sans

%\usepackage{microtype}

%%
% For nicely typeset tabular material
\usepackage{booktabs}

%%
% For graphics / images
\usepackage{graphicx}
\setkeys{Gin}{width=\linewidth,totalheight=\textheight,keepaspectratio}
\graphicspath{{res/}}

% The fancyvrb package lets us customize the formatting of verbatim
% environments.  We use a slightly smaller font.
\usepackage{fancyvrb}
\fvset{fontsize=\normalsize}

%%
% Prints argument within hanging parentheses (i.e., parentheses that take
% up no horizontal space).  Useful in tabular environments.
\newcommand{\hangp}[1]{\makebox[0pt][r]{(}#1\makebox[0pt][l]{)}}

%%
% Prints an asterisk that takes up no horizontal space.
% Useful in tabular environments.
\newcommand{\hangstar}{\makebox[0pt][l]{*}}

%%
% Prints a trailing space in a smart way.
\usepackage{xspace}

%%
% Math packages
\usepackage{braket}
\usepackage{amsmath}
\usepackage{amsthm}
\usepackage{mathtools}

%%
% Some shortcuts for Tufte's book titles.  The lowercase commands will
% produce the initials of the book title in italics.  The all-caps commands
% will print out the full title of the book in italics.
\newcommand{\vdqi}{\textit{VDQI}\xspace}
\newcommand{\ei}{\textit{EI}\xspace}
\newcommand{\ve}{\textit{VE}\xspace}
\newcommand{\be}{\textit{BE}\xspace}
\newcommand{\VDQI}{\textit{The Visual Display of Quantitative Information}\xspace}
\newcommand{\EI}{\textit{Envisioning Information}\xspace}
\newcommand{\VE}{\textit{Visual Explanations}\xspace}
\newcommand{\BE}{\textit{Beautiful Evidence}\xspace}

\newcommand{\TL}{Tufte-\LaTeX\xspace}

% Prints the month name (e.g., January) and the year (e.g., 2008)
\newcommand{\monthyear}{%
	\ifcase\month\or January\or February\or March\or April\or May\or June\or
		July\or August\or September\or October\or November\or
		December\fi\space\number\year
	}


	% Prints an epigraph and speaker in sans serif, all-caps type.
	\newcommand{\openepigraph}[2]{%
		%\sffamily\fontsize{14}{16}\selectfont
		\begin{fullwidth}
			\sffamily\large
			\begin{doublespace}
				\noindent\allcaps{#1}\\% epigraph
				\noindent\allcaps{#2}% author
			\end{doublespace}
		\end{fullwidth}
	}

	% Inserts a blank page
	\newcommand{\blankpage}{\newpage\hbox{}\thispagestyle{empty}\newpage}

	\usepackage{units}

	% Typesets the font size, leading, and measure in the form of 10/12x26 pc.
	\newcommand{\measure}[3]{#1/#2$\times$\unit[#3]{pc}}

	% Macros for typesetting the documentation
	\newcommand{\hlred}[1]{\textcolor{Maroon}{#1}}% prints in red
	\newcommand{\hangleft}[1]{\makebox[0pt][r]{#1}}
	\newcommand{\hairsp}{\hspace{1pt}}% hair space
	\newcommand{\hquad}{\hskip0.5em\relax}% half quad space
	\newcommand{\TODO}{\textcolor{red}{\bf TODO!}\xspace}
	\newcommand{\na}{\quad--}% used in tables for N/A cells
	\providecommand{\XeLaTeX}{X\lower.5ex\hbox{\kern-0.15em\reflectbox{E}}\kern-0.1em\LaTeX}
	\newcommand{\tXeLaTeX}{\XeLaTeX\index{XeLaTeX@\protect\XeLaTeX}}
	% \index{\texttt{\textbackslash xyz}@\hangleft{\texttt{\textbackslash}}\texttt{xyz}}
	\newcommand{\tuftebs}{\symbol{'134}}% a backslash in tt type in OT1/T1
	\newcommand{\doccmdnoindex}[2][]{\texttt{\tuftebs#2}}% command name -- adds backslash automatically (and doesn't add cmd to the index)
	\newcommand{\doccmddef}[2][]{%
		\hlred{\texttt{\tuftebs#2}}\label{cmd:#2}%
		\ifthenelse{\isempty{#1}}%
		{% add the command to the index
			\index{#2 command@\protect\hangleft{\texttt{\tuftebs}}\texttt{#2}}% command name
		}%
		{% add the command and package to the index
			\index{#2 command@\protect\hangleft{\texttt{\tuftebs}}\texttt{#2} (\texttt{#1} package)}% command name
			\index{#1 package@\texttt{#1} package}\index{packages!#1@\texttt{#1}}% package name
		}%
	}% command name -- adds backslash automatically
	\newcommand{\doccmd}[2][]{%
		\texttt{\tuftebs#2}%
		\ifthenelse{\isempty{#1}}%
		{% add the command to the index
			\index{#2 command@\protect\hangleft{\texttt{\tuftebs}}\texttt{#2}}% command name
		}%
		{% add the command and package to the index
			\index{#2 command@\protect\hangleft{\texttt{\tuftebs}}\texttt{#2} (\texttt{#1} package)}% command name
			\index{#1 package@\texttt{#1} package}\index{packages!#1@\texttt{#1}}% package name
		}%
	}% command name -- adds backslash automatically
	\newcommand{\docopt}[1]{\ensuremath{\langle}\textrm{\textit{#1}}\ensuremath{\rangle}}% optional command argument
	\newcommand{\docarg}[1]{\textrm{\textit{#1}}}% (required) command argument
	\newenvironment{docspec}{\begin{quotation}\ttfamily\parskip0pt\parindent0pt\ignorespaces}{\end{quotation}}% command specification environment
	\newcommand{\docenv}[1]{\texttt{#1}\index{#1 environment@\texttt{#1} environment}\index{environments!#1@\texttt{#1}}}% environment name
	\newcommand{\docenvdef}[1]{\hlred{\texttt{#1}}\label{env:#1}\index{#1 environment@\texttt{#1} environment}\index{environments!#1@\texttt{#1}}}% environment name
	\newcommand{\docpkg}[1]{\texttt{#1}\index{#1 package@\texttt{#1} package}\index{packages!#1@\texttt{#1}}}% package name
	\newcommand{\doccls}[1]{\texttt{#1}}% document class name
	\newcommand{\docclsopt}[1]{\texttt{#1}\index{#1 class option@\texttt{#1} class option}\index{class options!#1@\texttt{#1}}}% document class option name
	\newcommand{\docclsoptdef}[1]{\hlred{\texttt{#1}}\label{clsopt:#1}\index{#1 class option@\texttt{#1} class option}\index{class options!#1@\texttt{#1}}}% document class option name defined
	\newcommand{\docmsg}[2]{\bigskip\begin{fullwidth}\noindent\ttfamily#1\end{fullwidth}\medskip\par\noindent#2}
	\newcommand{\docfilehook}[2]{\texttt{#1}\index{file hooks!#2}\index{#1@\texttt{#1}}}
	\newcommand{\doccounter}[1]{\texttt{#1}\index{#1 counter@\texttt{#1} counter}}

	% MATH STUFF
	\newtheorem{axiom}{Axiom}
	\newtheorem{theorem}{Theorem}
	\newtheorem{definition}{Definition}
	\newtheorem{method}{Method}
	\DeclarePairedDelimiterX{\norm}[1]{\lVert}{\rVert}{#1}
	\newcommand{\commutator}[3][1.2em]{[ \makebox[#1]{$#2$} , \makebox[#1]{$#3$} ]}
	\newcommand{\infint}{\int^{\infty}_{-\infty}}
	\newcommand{\pfrac}[2]{\frac{\partial {#1} }{\partial {#2} }}
	\immediate\write18{bibtex \jobname}


	% Generates the index
	\usepackage{makeidx}
	\makeindex

	\begin{document}

% Front matter
\frontmatter

% r.1 blank page
\blankpage

% v.2 epigraphs



% r.3 full title page
\maketitle


% v.4 copyright page

% r.5 contents
\tableofcontents

\listoffigures

\listoftables

% r.7 dedication

% r.9 introduction
\cleardoublepage
\chapter*{Introduction}
This is based off notes from various classical mechanics books.

%%
% Start the main matter (normal chapters)
\mainmatter


\chapter{Variational Calculus}
Define the integral
\begin{equation}
	\label{eq:1}
	I = \int^b_a F(y,y',x)dx
\end{equation}

as a functional that acts on a curve $y(x)$. Its use is as a cost function, be it time or distance or some other measurable quantity. The curve is chosen to make the integral stationary (minimum or maximum). Hence consider

\begin{equation}
	\label{eq:2}
	y(x) \rightarrow y(x) + \alpha\nu(x)
\end{equation}

where the parameter $\alpha$ is small and $\nu(x)$ is an arbitrary function. We can redefine the stationary requirement as

\begin{equation}
	\label{eq:3}
	\left.\frac{dI}{d\alpha}\right|_{\alpha=0} = 0
\end{equation}

for any function. Substitute equation \ref{eq:2} into \ref{eq:1},

\begin{equation}
	\label{eq:4}
	I(y,\alpha) = \int^b_a F(y+\alpha\nu, y'+\alpha\nu',x)dx
\end{equation}

Now expand as a Taylor series in $\alpha$, only writing out up to first order terms

\begin{equation}
	\label{eq:5}
	I(y,\alpha) = \int^b_a\left(\pfrac{F}{y}\alpha\nu + \pfrac{F}{y'}\alpha\nu'\right)dx + O(\alpha^2)
\end{equation}

Set the first-order variations to zero

\begin{equation}
	\label{eq:6}
	\delta I = \int^b_a \left(\pfrac{F}{y}\alpha\nu + \pfrac{F}{y'}\alpha\nu'\right)dx = 0
\end{equation}

The second term can be dealt with by integration by parts

\begin{equation}
	\label{eq:7}
	\left[\nu\pfrac{F}{y'}\right]^b_a + \int^b_a\left[\pfrac{F}{y} - \frac{d}{dx}\left(\pfrac{F}{y'}\right)\right]\nu(x)dx = 0
\end{equation}

We can now apply the restriction that the end points are fixed and the function must pass through then so that $\nu(a) = \nu(b) = 0$. We also recall that it works for arbitrary functions. Therefore the Euler-Lagrange equation comes straight out

\begin{equation}
	\label{eq:el1}
	\pfrac{F}{y} = \frac{d}{dx}\left(\pfrac{F}{y'}\right)
\end{equation}

There are certain special cases, if $F$ is constant in one or more of the variables. If $F$ does not contain $y$ explicitly, the Euler-Lagrange equation trivially reduces to

\begin{equation}
	\label{eq:el2}
	\pfrac{F}{y'} = C
\end{equation}

If $F$ does not contain $x$ explicitly, multiply equation \ref{eq:el1} by $y'$

\begin{equation}
	\label{eq:8}
	y'\pfrac{F}{y} = y'\frac{d}{dx}\left(\pfrac{F}{y'}\right)
\end{equation}

Now write

\begin{equation}
	\label{eq:9}
	\frac{d}{dx}\left(y'\pfrac{F}{y'}\right) = y'\frac{d}{dx}\left(\pfrac{F}{y'}\right) + y''\pfrac{f}{y'}
\end{equation}

\begin{equation}
	\label{eq:10}
	y'\pfrac{F}{y} + y''\pfrac{F}{y'} = \frac{d}{dx}\left(y'\pfrac{F}{y'}\right)
\end{equation}

The left hand side of the equation is a total derivative of $F$ with respect to $x$, therefore we can integrate both sides to obtain

\begin{equation}
	\label{eq:el3}
	F = y'\pfrac{F}{y'} = C
\end{equation}

If we have several dependent variables $F = F(y_1, y_1', y_2, y_2', ..., y_n, y_n')$ where $y_i = y_i(x)$, the analysis changes to $n$ separate but simultaneous equations for each $y_i(x)$.

\begin{equation}
	\label{eq:el4}
	\pfrac{F}{y_i} = \frac{d}{dx}\left(\pfrac{F}{y'_i}\right)
\end{equation}

If we have independent variables $y = y(x_1, x_2, ..., x_n)$, the function becomes

\begin{equation}
	\label{eq:el5}
	\pfrac{F}{y} = \sum^n_{i=1}\left(\pfrac{F}{y_{x_i}}\right)
\end{equation}

where $y_{x_i} = \pfrac{y}{x_i}$. If we have higher order derivatives, $F = F(y, y', ..., y^{(n)}, x)$ then the Euler-Lagrange equation can be written

\begin{equation}
	\label{eq:el6}
	\pfrac{F}{y} - \frac{d}{dx}\left(\pfrac{F}{y'}\right) + \frac{d^2}{dx^2}\left(\pfrac{F}{y''}\right) - ... + (-1)^n \frac{d^n}{dx^n}\left(\pfrac{F}{y^{}(n)}\right) = 0
\end{equation}

%TODO Cover variable end points!

One can use the method of Lagrange undetermined multipliers to solve for constrained variation. If the constraint takes the form

\begin{equation}
	\label{eq:11}
	J = \int^b_a G(y, y', x)dx
\end{equation}

\section{Problems}

\subsection{1. Shortest curve joining two points}
\subsection{2. The brachistochrome}
\subsection{3. Fermat's principle}
\subsection{4. Total derivatives}

\chapter{Particles}
Before delving into the physical applications of variational calculus, it is useful to review our current knowledge of classical mechanics. Let $\vec{r}$ be the radius vector of a particle from some given origin and $\vec{v}$ its velocity, relative to this origin. The linear momentum of the particle is defined as

\begin{equation}
	\label{eq:12}
	\vec{v} = \frac{d\vec{r}}{dt}
\end{equation}

Newton formalised classical mechanics into three laws,

\begin{enumerate}
	\item In an inertial reference frame, an object in equilibrium either remains at rest or moves at a constant velocity.
	\item In an inertial reference frame, the net force acting on an object is related to the momentum of that object by
	      \begin{equation}
	      	\label{eq:NII}
	      	\vec{F} = \frac{d\vec{p}}{dt}
	      \end{equation}
	\item Each force as an opposing reaction force of equal magnitude.
\end{enumerate}

Newton's second law can also be written as

\begin{equation}
	\label{eq:13}
	\vec{F} = m\vec{a}
\end{equation}

where the mass is constant and a, the vector acceleration of the particle is defined as

\begin{equation}
	\label{eq:14}
	\vec{a} = \frac{d\vec{v}}{dt} = \frac{d^2\vec{r}}{dt^2}
\end{equation}

The equation of motion is typically a differential equation of second order. Newton's second law also implies a conservation of momentum when the total force acting on a system is equal to zero. Particles also have angular momentum defined as

\begin{equation}
	\label{eq:15}
	\vec{L} = \vec{r} \times \vec{p}
\end{equation}

about some origin reference point from which the vectors are defined. The torque or moment force is similarly defined

\begin{equation}
	\label{eq:16}
	\vec{N} = \vec{r} \times \vec{F}
\end{equation}

From equation \ref{eq:NII} and equation \ref{eq:12}

\begin{equation}
	\label{eq:17}
	\vec{N} = \vec{r} \times \frac{d}{dt}m\vec{v}
\end{equation}

Now consider the identity

\begin{equation}
	\label{eq:18}
	\frac{d}{dt}(\vec{r}\times m\vec{v}) = \vec{v}\times m\vec{v} + \vec{r}\times\frac{d}{dt}(m\vec{v}) = \vec{r}\times\frac{d}{dt}(m\vec{v})
\end{equation}

Therefore

\begin{equation}
	\label{eq:19}
	\vec{N} = \frac{d}{dt}(\vec{r} \times m\vec{v}) = \frac{d\vec{L}}{dt}
\end{equation}

The conservation of angular momentum follows, as long as the total torque on a system is zero, angular momentum is conserved. We can define the energy required to move a particle from point 1 to point 2 by some `work done',

\begin{equation}
	\label{eq:20}
	W_{12} = \int^2_1 \vec{F}\cdot d\vec{s}
\end{equation}

For constant mass this reduces to

\begin{equation}
	\label{eq:21}
	\int\vec{F}\cdot d\vec{s}
\end{equation}



\chapter{Lagrangian Formalism}


\section{Problems}
\subsection{General Method}
First pick a sensible coordinate system. Then write the Lagrangian in terms of the kinetic and potential energy components.

\begin{equation}
	L = T - U
\end{equation}

The kinetic energy can be calculated from the system by breaking down the system into components and then finding the velocity components for each object in the system. In cartesian components this becomes

\begin{equation}
	T = \frac{m}{2}\dot{x}^2+\dot{y}^2+\dot{z}^2
\end{equation}

Once the Lagrangian is found, one can use the Euler-Lagrange equations to find the equations of motion for the system.


%%
% The back matter contains appendices, bibliographies, indices, glossaries, etc.
\backmatter

\bibliography{1_quantum}
\bibliographystyle{plainnat}


\printindex

\end{document}
