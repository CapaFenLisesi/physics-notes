States time evolve according to the Schrodinger equation:

\begin{equation}
	\label{eq:tdse}
	i\hbar\frac{d}{dt}\ket{\psi(t)} = H\ket{\psi(t)}
\end{equation}

For a given Hamiltonian $H = T + U$, we convert the classical expression with an equivalent quantum one, substitute this in, and then solve. If H has no explicit time dependence, the equation

\begin{equation}
	\label{eq:s1}
	i\hbar\ket{\dot{\psi}} = H\ket{\psi}
\end{equation}

The approach is to find the eigenvalues and eigenvectors of the Hamiltonian and then construct the propagator $U(t)$ in terms of these. Once we have $U(t)$, we can write

\begin{equation}
	\label{eq:s2}
	\ket{\psi(t)} = U(t)\ket{\psi(0)}
\end{equation}

Since the Schrodinger equation is first order in time, initial state value is sufficient and the initial value of the derivative is not required. Hence we solve the time-independent Schrodinger equation (which gives us eigenkets and eigenvalues)

\begin{equation}
	\label{eq:tise}
	H\ket{E} = E\ket{E}
\end{equation}

And assuming we find these, expand the state

\begin{equation}
	\label{eq:s3}
	\ket{\psi(t)}
\end{equation}
